\documentclass[a4paper,14pt]{article} % тип документа
%\documentclass[14pt]{extreport}
\usepackage{extsizes} % Возможность сделать 14-й шрифт


\usepackage{geometry} % Простой способ задавать поля
\geometry{top=25mm}
\geometry{bottom=35mm}
\geometry{left=20mm}
\geometry{right=20mm}

\setcounter{section}{0}

%%%Библиотеки
%\usepackage[warn]{mathtext}
%\usepackage[T2A]{fontenc} % кодировка
\usepackage[utf8]{inputenc} % кодировка исходного текста
\usepackage[english,russian]{babel} % локализация и переносы
\usepackage{caption}
\usepackage{listings}
\usepackage{amsmath,amsfonts,amssymb,amsthm,mathtools}
\usepackage{wasysym}
\usepackage{graphicx}%Вставка картинок правильная
\usepackage{float}%"Плавающие" картинки
\usepackage{wrapfig}%Обтекание фигур (таблиц, картинок и прочего)
\usepackage{fancyhdr} %загрузим пакет
\usepackage{lscape}
\usepackage{xcolor}
\usepackage{dsfont}
%\usepackage{indentfirst}
\usepackage[normalem]{ulem}
\usepackage{hyperref}




%%% DRAGON STUFF
\usepackage{scalerel}
\usepackage{mathtools}

\DeclareMathOperator*{\myint}{\ThisStyle{\rotatebox{25}{$\SavedStyle\!\int\!\!\!$}}}

\DeclareMathOperator*{\myoint}{\ThisStyle{\rotatebox{25}{$\SavedStyle\!\oint\!\!\!$}}}

\usepackage{scalerel}
\usepackage{graphicx}
%%% END 

%%%Конец библиотек

%%%Настройка ссылок
\hypersetup
{
colorlinks=true,
linkcolor=blue,
filecolor=magenta,
urlcolor=blue
}
%%%Конец настройки ссылок


%%%Настройка колонтитулы
	\pagestyle{fancy}
	\fancyhead{}
	\fancyhead[L]{Домашнее задание}
	\fancyhead[R]{Крейнин Матвей, группа Б05-005}
	\fancyfoot{}
    \fancyfoot[C]{\thepage}
    \fancyfoot[R]{ОВАиТК}
%%%конец настройки колонтитулы



\begin{document}
%%%%Начало документа%%%%

\section{Задание 2}
\subsection{Задача 1}
Докажите, что если в моноиде M у всякого элемента существует левый (не обязательно единственный) обратный, т.е.

\begin{equation*}
	\forall a \in M \exists a^{-1} : a^{-1} \cdot a = e 
\end{equation*}
то M является группой.

Моноид является по определению полугруппой и у него есть ещё единственный нейтральный элемент.
А группа это моноид плюс к этому существование обратного элемента. Отличие в начальных условиях только в том, что у нас может существовать несколько обратных.

Тогда у нас существует обратный элемент для обратного элемента $a$. Т.е. $\exists a^{*} : a^{*} \cdot a^{-1} = e$.

Тогда получим: 
\begin{equation*}
	a^{-1} \cdot a = e \rightarrow a^{*} \cdot a^{-1} \cdot a = a^{*} \cdot e \rightarrow e \cdot a = a^{*} e \rightarrow a = a^{*}
\end{equation*}

Теперь допустим, что $a^{-1}$ не единственный обратный элемент к элементу a, т.е. $\exists a^{\#} : a^{\#} \cdot a = a \cdot a^{\#} = e, a^{-1} \not= a^{\#}$

Тогда получим:
\begin{equation*}
	a^{\#} \cdot a = a \cdot a^{\#} \rightarrow a^{\#} \cdot a \cdot a^{-1} = e \cdot a^{-1} \rightarrow a^{\#} \cdot e = e \cdot a^{-1} \rightarrow a^{\#} = a^{-1}.
\end{equation*}
Пришли к противоречию и получается, что обратный элемент единственный, тогда моноид $M$ полностью удовлетворяет опеделению группы.


\subsection{Задача 2}
Постройте таблицу Кэли для группы G порядка 4, в которой $\forall g \in G g^2 = e$. Приведите явный пример такой группы.

\begin{table}[H]
	\begin{tabular}{|l|l|l|l|l|}
	\hline
	$\cdot$ & e & a & b & c \\ \hline
	e   & e & a & b & c \\ \hline
	a   & a & e & c & b \\ \hline
	b   & b & c & e & a \\ \hline
	c   & c & b & a & e \\ \hline
	\end{tabular}
	\end{table}

Пусть $e = 0, a = 01, b = 10, c = 11$, а за операцию $\cdot$ возьмём XOR, тогда получим таблицу:

\begin{table}[H]
	\begin{tabular}{|l|l|l|l|l|}
	\hline
	XOR & 0  & 01 & 10 & 11 \\ \hline
	0   & 0  & 01 & 10 & 11 \\ \hline
	01  & 01 & 0  & 11 & 10 \\ \hline
	10  & 10 & 11 & 0  & 01 \\ \hline
	11  & 11 & 10 & 01 & 0  \\ \hline
	\end{tabular}
	\end{table}

\subsection{Задача 3}
Проверим первое свойство группы для чисел $a = \frac{1}{2}, b = \frac{1}{3}, c = \frac{3}{4}$

\begin{equation*}
	a \odot b = \{ a - b\} = \{\frac{1}{2} - \frac{1}{3}\} = \frac{1}{6} - \lfloor \frac{1}{6} \rfloor = \frac{1}{6} - (0) = \frac{1}{6}
\end{equation*}

\begin{equation*}
	(a \odot b) \odot c = \{\frac{1}{6} - \frac{3}{4}\} = -\frac{7}{12} - \lfloor -\frac{7}{12} \rfloor = -\frac{7}{12} - (-1) = -\frac{5}{12}
\end{equation*}

\begin{equation*}
	b \odot  c = \{b - c\} = \{\frac{1}{3} - \frac{3}{4}\} = -\frac{5}{12} - \lfloor -\frac{5}{12} \rfloor = -\frac{5}{12} - (-1) = \frac{7}{12}
\end{equation*}

\begin{equation*}
	a \odot (b \odot c) = \{\frac{1}{2} - \frac{7}{12}\} = -\frac{1}{12} - \lfloor -\frac{1}{12} \rfloor = \frac{11}{12}
\end{equation*}

Но $\frac{11}{12} \not = -\frac{5}{12}$, сл-но это не группа.


\subsection{Задача 4}
Пусть $e = 0, a = 01, b = 10, c = 11$, а за операцию $\cdot$ возьмём XOR, тогда получим таблицу:

\begin{table}[H]
	\begin{tabular}{|l|l|l|l|l|}
	\hline
	XOR & 0  & 01 & 10 & 11 \\ \hline
	0   & 0  & 01 & 10 & 11 \\ \hline
	01  & 01 & 0  & 11 & 10 \\ \hline
	10  & 10 & 11 & 0  & 01 \\ \hline
	11  & 11 & 10 & 01 & 0  \\ \hline
	\end{tabular}
\end{table}

Меньшего порядка нельзя, исходя из того, что у нас должно быть как минимум 4 элемента в группе, чтобы было 4 решения уравнения.
\end{document}