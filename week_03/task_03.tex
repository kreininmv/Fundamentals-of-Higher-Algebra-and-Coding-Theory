\documentclass[a4paper,14pt]{article} % тип документа
%\documentclass[14pt]{extreport}
\usepackage{extsizes} % Возможность сделать 14-й шрифт


\usepackage{geometry} % Простой способ задавать поля
\geometry{top=25mm}
\geometry{bottom=35mm}
\geometry{left=20mm}
\geometry{right=20mm}

\setcounter{section}{0}

%%%Библиотеки
%\usepackage[warn]{mathtext}
%\usepackage[T2A]{fontenc} % кодировка
\usepackage[utf8]{inputenc} % кодировка исходного текста
\usepackage[english,russian]{babel} % локализация и переносы
\usepackage{caption}
\usepackage{listings}
\usepackage{amsmath,amsfonts,amssymb,amsthm,mathtools}
\usepackage{wasysym}
\usepackage{graphicx}%Вставка картинок правильная
\usepackage{float}%"Плавающие" картинки
\usepackage{wrapfig}%Обтекание фигур (таблиц, картинок и прочего)
\usepackage{fancyhdr} %загрузим пакет
\usepackage{lscape}
\usepackage{xcolor}
\usepackage{dsfont}
%\usepackage{indentfirst}
\usepackage[normalem]{ulem}
\usepackage{hyperref}




%%% DRAGON STUFF
\usepackage{scalerel}
\usepackage{mathtools}

\DeclareMathOperator*{\myint}{\ThisStyle{\rotatebox{25}{$\SavedStyle\!\int\!\!\!$}}}

\DeclareMathOperator*{\myoint}{\ThisStyle{\rotatebox{25}{$\SavedStyle\!\oint\!\!\!$}}}

\usepackage{scalerel}
\usepackage{graphicx}
%%% END 

%%%Конец библиотек

%%%Настройка ссылок
\hypersetup
{
colorlinks=true,
linkcolor=blue,
filecolor=magenta,
urlcolor=blue
}
%%%Конец настройки ссылок


%%%Настройка колонтитулы
	\pagestyle{fancy}
	\fancyhead{}
	\fancyhead[L]{Домашнее задание}
	\fancyhead[R]{Крейнин Матвей, группа Б05-005}
	\fancyfoot{}
    \fancyfoot[C]{\thepage}
    \fancyfoot[R]{ОВАиТК}
%%%конец настройки колонтитулы



\begin{document}
%%%%Начало документа%%%%

\section{Задание 3}
\subsection{Задача 1}
Докажите: $\text{НОД(k, n)} = 1 \leftrightarrow \exists t : t \cdot k \equiv 1 mod n$

$\leftarrow :$ пусть d --  общий делитель k и n и известно, что $tk \equiv 1 mod n$. Тогда $tk = nm +1, tk - mn = 1$, следовательно $d = 1$, то есть единственный общий делитель 1, то есть числа взаимно простые.

$\rightarrow :$ Пусть НОД(k, n) $ = 1$. Рассмотрим вычеты, кратые $[k]_n$, из множество $(k, n) = \{z : xk + yn, x, y \in \mathds{Z} \} = \text{НОД(k, n)} \mathds{Z} = \mathds{Z}$, значит $\exists x : kx = 1 - ny$, что и требовалось доказать.

\subsection{Задача 2}
Вычислите $17^{668}$ mod $27$

$17^{668} = 17^{4 \cdot 167} = (17^{4})^{167} = 10^{167} = | 10^6$ mod $27 = 1 | = 10^5 \cdot (1)^27 = 10^5 = 19$ 

\underline{\textbf{Ответ:}} 19.

\subsection{Задача 3}
Вычислите $2^{21^{42069}}$ mod $14$

\begin{equation*}
	2^{3+1} \equiv 2 \text{ mod } 14 \rightarrowtail 2^{3+1}\cdot 2^3 \equiv 2 \text{ mod } 14
\end{equation*}
Тогда получаем:
\begin{equation*}
	2^{3n+1} \equiv 2 \text{ mod } 14 
\end{equation*}
\begin{equation*}
	2^{3n} \equiv 8 \text{ mod } 14 \rightarrowtail 2^{21^{42069}} \equiv 8 \text{ mod } 14
\end{equation*}

\underline{\textbf{Ответ: }} 8

\subsection{Задача 4}
Изоморфны ли группы:
\begin{enumerate}
	\item $C_{13} \times C_{13}$ и $C_{169}$
	\item $C_7 \times C_{15}$ и $C_5 \times C_{21}$
\end{enumerate}

\begin{enumerate}
	\item В группе $C_{13} \times C_{13}$ порядки всех элементов не превосходят 13, а в группе $C_{169}$ есть элемент порядка 169. Изоморфизм сохраняет порядки элементов, поэтому группы неиозоморфны.
	\item Воспользуемся Китайской теоремой об остатках: $C_7 \times C_{15} = C_7 \times C_3 \times C_5$, $C_5 \times C_{21} = C_5 \times C_3 \times C_7$, откуда получим, что $C_7 \times C_3 \times C_5 = C_5 \times C_3 \times C_7$ с точностью до перестановок.
\end{enumerate}

{\textbf{Ответ:}} 1) Нет 2) Да

\subsubsection{Задача 7}
Порождают ли перестановки порядка 3 группу $S_{33}$?

Порядок перестановки -- это НОК длин циклов в цикловом разложении. Т.к. $3 = 3 \cdot 1$, то циклы либо длины 1, либо длины 3. Циклы длины 3 -- чётные перестановки (т.к. любая четная перестановка является произведением циклов длины 3). Поэтому их произведение -- чётная перестановка $\rightarrow$ они образуют подгруппу четных перестановок.

\underline{\textbf{Ответ:}} Нет.




\end{document}