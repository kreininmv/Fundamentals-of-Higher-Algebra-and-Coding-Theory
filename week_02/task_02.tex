\documentclass[a4paper,14pt]{article} % тип документа
%\documentclass[14pt]{extreport}
\usepackage{extsizes} % Возможность сделать 14-й шрифт


\usepackage{geometry} % Простой способ задавать поля
\geometry{top=25mm}
\geometry{bottom=35mm}
\geometry{left=20mm}
\geometry{right=20mm}

\setcounter{section}{0}

%%%Библиотеки
%\usepackage[warn]{mathtext}
%\usepackage[T2A]{fontenc} % кодировка
\usepackage[utf8]{inputenc} % кодировка исходного текста
\usepackage[english,russian]{babel} % локализация и переносы
\usepackage{caption}
\usepackage{listings}
\usepackage{amsmath,amsfonts,amssymb,amsthm,mathtools}
\usepackage{wasysym}
\usepackage{graphicx}%Вставка картинок правильная
\usepackage{float}%"Плавающие" картинки
\usepackage{wrapfig}%Обтекание фигур (таблиц, картинок и прочего)
\usepackage{fancyhdr} %загрузим пакет
\usepackage{lscape}
\usepackage{xcolor}
\usepackage{dsfont}
%\usepackage{indentfirst}
\usepackage[normalem]{ulem}
\usepackage{hyperref}




%%% DRAGON STUFF
\usepackage{scalerel}
\usepackage{mathtools}

\DeclareMathOperator*{\myint}{\ThisStyle{\rotatebox{25}{$\SavedStyle\!\int\!\!\!$}}}

\DeclareMathOperator*{\myoint}{\ThisStyle{\rotatebox{25}{$\SavedStyle\!\oint\!\!\!$}}}

\usepackage{scalerel}
\usepackage{graphicx}
%%% END 

%%%Конец библиотек

%%%Настройка ссылок
\hypersetup
{
colorlinks=true,
linkcolor=blue,
filecolor=magenta,
urlcolor=blue
}
%%%Конец настройки ссылок


%%%Настройка колонтитулы
	\pagestyle{fancy}
	\fancyhead{}
	\fancyhead[L]{Домашнее задание}
	\fancyhead[R]{Крейнин Матвей, группа Б05-005}
	\fancyfoot{}
    \fancyfoot[C]{\thepage}
    \fancyfoot[R]{ОВАиТК}
%%%конец настройки колонтитулы



\begin{document}
%%%%Начало документа%%%%

\section{Задание 2}
\subsection{Задача 1}
Докажите, что в циклической группе конечного порядка всякая подгруппа является циклической.

Пусть $H < G$, где G -- циклическая группа с порождающим элементом a, и $b \in H : b = a^n$, при чём b такое, что n -- наименьшая степень, при которой $a^n$ входит в подгруппу.
Тогда $\forall c \in H$ можно записать: $c = a^m$, $m = n \cdot d + r, (m > n)$, но тогда и $a^r$ будет входить в подгруппу, т.к. $a^q = e$, (q -- порядок элемента).
$a^{q-n} = b^{-1} \in H \longrightarrow (b^{-1})^d \cdot c = a^{dq - dn} \cdot a^{nd+r} = a^{dq} \cdot a^r = a^r \in H$, получили противоречие, 
т.к. $0 \leq r < n$, а n -- по условию минимальная степень с таким свойством, сл-но $r = 0$, 
но это и будет означать, что подгруппа циклическая, т.к. $\forall c \in H \exists d : b^d = e$.

\subsection{Задача 2}
Порядок элемента a равен d (конечное число). Найти порядок элемента $a^k$.

По определению $a^d = e$, где d -- наименьшее такое число.
Тогда $(a^k)^q = e$, где q -- искомый порядок, то есть $a^{kq} = e \longrightarrow d | kq$, 
где q -- наименьшее, получаем, что $q = \frac{\text{НОК(d, k)}}{k}$.

\underline{\textbf{Ответ:}} $q = \frac{\text{НОК(d, k)}}{k}$.

\subsection{Задача 3}
Найдите все подгруппы циклической группы порядка n.

Пусть a -- порождающий элемент G и $H < G$. $a^n = e$, e должна будет принадлежать любой подгруппе,
следовательно $q | n, b = a^q$, где b -- порождающий элемент подгруппы, то есть число подгрупп не можеть быть больше количества делителей n.
Но и меньше тоже не может, потому что каждый делитель порождает подгруппу, и эти подгруппы не совпадают.

\underline{\textbf{Ответ:}} Все делители числа n.


\subsubsection*{1.} Сколько различных решений имеет уравнение $x^k = e$ в группе $C_m$, 
\newline где $k, m \in \mathds{N}$?

Пусть $a^y = x$ (т.к. циклическая группа), тогда уравнение принимает вид: $a^{yk} = e = a^m$, то есть уравнение равносильно следующему:
$m|yk$, будем искать кол-во решений при условии, что $y < m$ (т.е. решения разные). y должен делиться на $\frac{\text{НОК(m, k)}}{k}$, чтобы
выполнялось первое условие. При наложении второго получим кол-во решений для y равное $\frac{km}{\text{НОК(m, k)}} = \text{НОД(k, m)}$

\underline{\textbf{Ответ: }} $\text{НОД(k, m)}$.

\subsubsection*{2. } Найдите все элементы порядка 10 в группе $C_100$.

По прошлой задаче решений будет НОД(100, 10) $ = 10$, вот они слева на право:
\begin{equation*}
a^{10}, a^{20}, a^{30}, a^{40}, a^{50}, a^{60}, a^{70}, a^{80}, a^{90}, a^{100}.
\end{equation*}
где a - порождающий элемент, $a^100 = e$ и $(a^p)^{10}$ -- искомое.

\underline{\textbf{Ответ:}} $\{a^{10}, a^{20}, a^{30}, a^{40}, a^{50}, a^{60}, a^{70}, a^{80}, a^{90}, a^{100}.\}$


\end{document}