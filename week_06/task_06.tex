\documentclass[a4paper,14pt]{article} % тип документа
%\documentclass[14pt]{extreport}
\usepackage{extsizes} % Возможность сделать 14-й шрифт


\usepackage{geometry} % Простой способ задавать поля
\geometry{top=25mm}
\geometry{bottom=35mm}
\geometry{left=20mm}
\geometry{right=20mm}

\setcounter{section}{0}

%%%Библиотеки
%\usepackage[warn]{mathtext}
%\usepackage[T2A]{fontenc} % кодировка
\usepackage[utf8]{inputenc} % кодировка исходного текста
\usepackage[english,russian]{babel} % локализация и переносы
\usepackage{caption}
\usepackage{listings}
\usepackage{amsmath,amsfonts,amssymb,amsthm,mathtools}
\usepackage{wasysym}
\usepackage{graphicx}%Вставка картинок правильная
\usepackage{float}%"Плавающие" картинки
\usepackage{wrapfig}%Обтекание фигур (таблиц, картинок и прочего)
\usepackage{fancyhdr} %загрузим пакет
\usepackage{lscape}
\usepackage{xcolor}
\usepackage{dsfont}
%\usepackage{indentfirst}
\usepackage[normalem]{ulem}
\usepackage{hyperref}




%%% DRAGON STUFF
\usepackage{scalerel}
\usepackage{mathtools}

\DeclareMathOperator*{\myint}{\ThisStyle{\rotatebox{25}{$\SavedStyle\!\int\!\!\!$}}}

\DeclareMathOperator*{\myoint}{\ThisStyle{\rotatebox{25}{$\SavedStyle\!\oint\!\!\!$}}}

\usepackage{scalerel}
\usepackage{graphicx}
%%% END 

%%%Конец библиотек

%%%Настройка ссылок
\hypersetup { colorlinks=true, linkcolor=blue, filecolor=magenta, urlcolor=blue }
%%%Конец настройки ссылок

%%%Настройка колонтитулы
	\pagestyle{fancy}
	\fancyhead{}
	\fancyhead[L]{Домашнее задание}
	\fancyhead[R]{Крейнин Матвей, группа Б05-005}
	\fancyfoot{}
    \fancyfoot[C]{\thepage}
    \fancyfoot[R]{ОВАиТК}
%%%конец настройки колонтитулы

\begin{document}
%%%%Начало документа%%%%
\section{Задание №6}
\subsection{Задача №1}
Приведем пример кольца, не являющегося циклической группой по сложению: \\\\
Рассмотрим кольцо рациональных чисел с операциями сложения и умножения. В первую очередь обратим внимание, что это кольцо, так как множество замкнуто относительно операций, также оно является коммутативной группой относительно сложения с нейтральным элементом $0$ (обратным для любого элемнта $x$ является $-1\cdot x$). Умножение ассоциативно, и в силу свойств операций выполняется дистрибутивность.\\\\
Предположим, что в кольце существует элемент $a$ такой, что $\forall b \in \langle \mathbb{Q,~+,~\cdot} \rangle~\exists~k \in \mathbb{Z}:~b = ka$, то есть кольцо является циклической группой по сложению, тогда заметим, что элемент $\frac{1}{5}a \in \langle \mathbb{Q,~+,~\cdot} \rangle$, но не существует $k \in \mathbb{Z}:~ka = \frac{1}{5}a$. Противоречие, значит пример корректен. 
\subsection{Задача №2}
Докажем, что в коммутативном кольце сумма нильпотентных элементов — тоже нильпотентный элемент. А также приведем пример некоммутативного кольца, где это не выполняется.\\\\
Элемент кольца $a$ называется нильпотентным, если существует $n>0$ такое, что $a^n = 0$. Рассмотрим два нильпотентных элемента $a~и~b$ кольца $M$ с операциями $\langle+,~\cdot ~\rangle$. По определению $\exists~n,~m>0:~a^n = 0,~b^m = 0$. Если же рассмотреть сумму данных элементов в степени $mn$ и разложить ее в бином Ньютона (это справедливо, так как кольцо по условию коммутативное): $$(a+b)^{mn} = \sum\limits_{k = 1}^{mn}C_{mn}^{k}a^{mn - k}b^{k}$$
Видим, что слагаемые при $k\geq m$ равны $0$, так как $b^k = 0$. В случае же $k < m$ (будем считать, что $m,n>1$, иначе тривиальный случай, когда $a = 0~\text{или}~b = 0$ и утверждение подавно выполняется) получаем, что $mn-k>mn - m = m(n-1) \geq n$, следовательно, $a^{mn - k} = 0$, то есть $(a+b)^{mn} = 0$.\\\\
В качестве некоммутативного кольца возьмем матрицы 2 на 2: $M_{2\times2}$\\
Рассмотрим $A = 
\begin{pmatrix}
  0& 1\\
  0& 0
\end{pmatrix}, 
B = \begin{pmatrix}
  0& 0\\
  1& 0
\end{pmatrix}$ $\Rightarrow$ $A^2 = B^2 =$ $\begin{pmatrix}
  0& 0\\
  0& 0
\end{pmatrix}$ \\
При этом матрица $A+B=$ $\begin{pmatrix}
  0& 1\\
  1& 0
\end{pmatrix}$, которая в нечетных степенях равна себе же, а в четных степенях является единичной матрицей. Пример корректен из построения.
\subsection{Задача №3}
Найдем все нильпотентные элементы в кольце $\mathbb{Z}/(72)$.\\\\
Мы знаем, что множество элементов данного кольца представимо в виде $\overline{0}, \overline{1}, \overline{2}, \dots, \overline{71}$. Рассмотрим произвольный элемент $a \in \mathbb{Z} / (72)$. Условие $a^k = \overline{0}$ равносильно делимости $a^k$ на $72$, отсюда следует, что нильпотентные вычеты "--- это те вычеты, которые дают остатки, делящиеся на все простые числа в разложении $72 = 2^3 \cdot 3^2$, то есть 2, и 3 или другими словами кратные 6. Нетрудно понять, что таковыми будут числа \[ \overline{0},~\overline{6},~\overline{12},~ \overline{18},~ \overline{24}, ~\overline{30}, ~\overline{36}, ~\overline{42}, ~\overline{48},~ \overline{54},~ \overline{60},~ \overline{66}. \]

  \noindent\textbf{Ответ:} $\overline{6k}$, где $k = \overline{0,11}$.
  \subsection{Задача №4}
\textbf{1)}
      Докажем, что в кольце с единицей из нильпотентности $x$ следует обратимость $1-x$, то есть существует левый и правый обратные элементы (левый и правый один и тот же из определения с лекции). Из определения нильпотентного элемента существует $n:~x^n = 0$. Тогда, используя дистрибутивность, получаем
      $$(1 - x)(1 + x + x^2 + \ldots + x^{n-1}) = (1 + x + x^2 + \ldots + x^{n-1}) - (x + x^2 + \ldots + x^n) = 1 - x^n = 1$$
      Отсюда следует существование правого обратного элемента. Аналогичным образом при перестановке множителей множителей получаем левый обратный
      (умножение степеней $x$ коммутативно). Следовательно, $1-x$ обратимый.\\\\
\textbf{2)}
      Покажем, что в любом кольце $R$ с единицей обратимые элементы образуют группу по умножению. Замкнутость: произведение двух обратимых элементов является обратимым элементом, поскольку
      если $a, b \in R$ обратимы, то для них существуют обратные элементы $a^{-1}, b^{-1} \in R$,
      а потому
      $$ a \cdot b \cdot b^{-1} \cdot a^{-1} = a \cdot a^{-1} = 1 ~~ (a \cdot b)^{-1} = b^{-1} \cdot a^{-1} $$
      То есть присутствует замкнутость относительно операции. Ассоциативность операции следует из определения кольца.
      Очевидно, единица является нейтральным элементом.
      Ну и наконец, если $a \in R$ обратим, то, по определению, существует $с \in R$ такой, что
      $aс = сa = 1$, при этом, как можно заметить, $с$ тоже является обратимым.\\\\
\textbf{3)}
      Найдем порядок группы обратимых элементов в кольце $\mathbb{Z} / (12)$ и проверим, является
      ли она циклической. Обратимыми элементами в данном кольце являются те остатки по модулю 12,
      которые при умножении дают остаток 1 по тому же модулю. Таковыми являются числа 1, 5, 7 и 11:
      \begin{align*}
        1 \cdot 1 = 1 &\equiv 1 \mod{12},
        \\
        5 \cdot 5 = 25 &\equiv 1 \mod{12},
        \\
        7 \cdot 7 = 49 &\equiv 1 \mod{12},
        \\
        11 \cdot 11 = 121 &\equiv 1 \mod{12}.
      \end{align*}
      Группа конечная, ее порядок равен 4. При этом нетрудно видеть, что она не
      является циклической, поскольку любой элемент группы во второй степени уже
      равен единице.\\
\textbf{Ответ:} порядок группы равен 4, она не является циклической.

\vspace{1 cm}
\small
\textbf{Происходили обсуждения с Никитой и Даней по ходу всей домашки}
\end{document}