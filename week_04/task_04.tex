\documentclass[a4paper,14pt]{article} % тип документа
%\documentclass[14pt]{extreport}
\usepackage{extsizes} % Возможность сделать 14-й шрифт


\usepackage{geometry} % Простой способ задавать поля
\geometry{top=25mm}
\geometry{bottom=35mm}
\geometry{left=20mm}
\geometry{right=20mm}

\setcounter{section}{0}

%%%Библиотеки
%\usepackage[warn]{mathtext}
%\usepackage[T2A]{fontenc} % кодировка
\usepackage[utf8]{inputenc} % кодировка исходного текста
\usepackage[english,russian]{babel} % локализация и переносы
\usepackage{caption}
\usepackage{listings}
\usepackage{amsmath,amsfonts,amssymb,amsthm,mathtools}
\usepackage{wasysym}
\usepackage{graphicx}%Вставка картинок правильная
\usepackage{float}%"Плавающие" картинки
\usepackage{wrapfig}%Обтекание фигур (таблиц, картинок и прочего)
\usepackage{fancyhdr} %загрузим пакет
\usepackage{lscape}
\usepackage{xcolor}
\usepackage{dsfont}
%\usepackage{indentfirst}
\usepackage[normalem]{ulem}
\usepackage{hyperref}




%%% DRAGON STUFF
\usepackage{scalerel}
\usepackage{mathtools}

\DeclareMathOperator*{\myint}{\ThisStyle{\rotatebox{25}{$\SavedStyle\!\int\!\!\!$}}}

\DeclareMathOperator*{\myoint}{\ThisStyle{\rotatebox{25}{$\SavedStyle\!\oint\!\!\!$}}}

\usepackage{scalerel}
\usepackage{graphicx}
%%% END 

%%%Конец библиотек

%%%Настройка ссылок
\hypersetup
{
colorlinks=true,
linkcolor=blue,
filecolor=magenta,
urlcolor=blue
}
%%%Конец настройки ссылок


%%%Настройка колонтитулы
	\pagestyle{fancy}
	\fancyhead{}
	\fancyhead[L]{Домашнее задание}
	\fancyhead[R]{Крейнин Матвей, группа Б05-005}
	\fancyfoot{}
    \fancyfoot[C]{\thepage}
    \fancyfoot[R]{ОВАиТК}
%%%конец настройки колонтитулы



\begin{document}
%%%%Начало документа%%%%

\section{Задание 4}
\subsection{Задача 1}
\textbf{Доказательство:} будем доказывать по индукции:

\begin{enumerate}
	\item база $m = 1$ очевидна из условия, т.к. $ab \in N$
	\item предположим, что верно для $m = n \in \mathds{N}$, т.е $a^nb^b \in N$
	\item тогда шагом индукции будет доказательство принадлежности к N элемента $a^{n+1}b^{n+1}$. 
	Используем лемму: два элемента группы $g_1$ и $g_2$ принадлежат одному классу смежности по подгруппе $H$ тогда и только тогда, когда $g_1^{-1}g_2 \in H$.
	Сама подгруппа также является классом смежности, тогда элементы $a^nb^n$ и $a^{n+1}b^{n+1}$ принадлежат одному и тому же классу смежности, т.е. N, т.к. $a^nb^n \in N$,
	если $(a^nb^n)^{-1}a^{n+1}b^{n+1} \in N$. Совершим преобразования: $(a^nb^n)^{-1}a^{n+1}b^{n+1} = b^{-n}a^{-n}a^{n+1}b^{n+1} = b^{-n}abb^{n}$
	Т.к. $N$ -- нормальная подгруппа, то для $b^{-n} \in G$ выполняется $b^{-n}N = Nb^{-n}$, тогда для $ab \in N \exists h \in N : b^{-n}ab = hb^{-n}$. Воспользуемся данным равенством
	и подставим его в ранее полученное выражение: $b^{-n}abb^{n} = h b^{-n} b^{n} = h \in N$. Следовательно из критерия получаем, что $a^{n+1}b^{n+1} \in N$, т.е. утверждение доказано.
\end{enumerate}

Из определения нормальной группы следует, т.к. $a \in G, ab \in N \rightarrow a^{-1}aba = ba \in N$, тогда в точности аналогично пункту доказывается, что $\forall m \in \mathds{N} \rightarrow b^ma^m \in N$

Теперь заметим, тчо из определения подгруппы $a^0b^0 = b^0a^0 = e \in N$. Что касается целых отрицательных степеней, всё довольно очевидно, по определению подгруппы, если элемент лежит в ней, то и обратный элемент лежит.
Тогда остается воспользоваться тем, что $\forall m \in \mathds{N} (a^mb^m)^{-1} = b^{-m}a^{-m} \in N$ и $\forall m \in \mathds{N} (b^ma^m)^{-1} = a^{-m}b^{-m} \in N$. Таким образом, мы доказывали, что $\forall m \in \mathds{Z} \rightarrow a^mb^m, b^ma^m \in N$

\subsection{Задача 2}
Рассмотрим симметрическую группу $S_3$, перестановок из 3-х элементов. В качестве примера группы, которая не будет являться нормальной будет $F = \{e$, (1 2) $\}$, очевидно, что $H < S_3$. Докажем, что не является нормальной: $\exists x =$ (1 3) $\in S_3$.
Тогда $xF =$ $\{$ (1 3), (1 3)(1 2) $\}$, $Fx = \{$(1 3), (1 2)(1 3) $\}$, учитывая, что (1 3)(1 2) $=$ (1 2 3) $\not =$ (1 2)(1 3) $=$ (1 3 2), т.е. $xF \not = Fx$. Т.е. такая группа подходит.

Покажем, используя данный пример, что множество левых смежных классов с операцией, как у факторгруппы не образует группу. Рассмотрим левые смеэные классы (2 3) $F$ и (1 3 2)$F$, тогда получим, что (2 3)$F = \{$(2 3), (1 3 2) $\}$, (1 3 2)$F = \{$(1 3 2), (2 3)$\}$, т.е (2 3)$F =$(1 3 2)F.
Рассмотрим операцию, определенную на множестве левых смежных классов: (2 3)$F *$ (2 3)$F = F$, при этом в силу равенства (2 3)$H = $(1 3 2)$H$ можно записать то же произведение классов, как (1 3 2)$F * $(2 3)$F =$(1 3 2)(2 3)$F = $(1 3)F$ = \{$ (1 3), (1 2 3)$\} \not = F$, получаем, что определить 
групповую операцию на множестве левых смежных классов по подгруппе F не получится, и группой данное множество с такой операций не будет являться.

\subsection{Задача 3}
В силу операции сложения, определенной на рассматриваемой группе, получим, что $\mathds{Z}^3$ -- абелева. Тогда её подгруппа $F = \{$(x, y, z) | $x + y + z = 0 \}$ -- нормальная.
Отсюда по определению следует, что определена факторгруппа $\mathds{Z}^3 / H$. Докажем, тчо $\mathds{Z}^3 / H = \{M_k | \forall(x, y, z) \in M_k \rightarrow x + y + z = k, k \in \mathds{Z} \}$.

Сначала будем доказывать, что любые элементы, обладающие одним и тем же свойством, т.е. элементы $M_k = \{(x, y, z) | x + y + z = k \}$ лежат в одном классе смежности, воспользуемся леммой 2, т.е. $\forall g_1, g_2 \in M_k \rightarrow (-g_1) + g_2 = (x_2 - x_1, y_2 - y_1, z_2 - z_1)$,
остается заметить, что сумма компонент полученного вектора есть 0, сл-но, $(-g_1) + g_2 \in H$, по критерию элементы будут лежать в одном классе.

Теперь докажем, что если элементы лежат в разных множествах, то они лежат в разных классах, пользоваться будем той же леммой 2, т.е. будем рассматривать $M_k$ и $M_n$, где $n \not = k$, тогда возьмём произвольные элементы из $M_k$ и $M_n$, соотвественно, $g_k, g_n \rightarrow (-g_k) + g_n = (x_n - x_k, y_n - y_k, z_n - z_k)$, 
сумма компонент этого вектора: $n - k \not = 0$, полуачем, что $(-g_k) + g_n \not \in H$, по критерию элементы лежат в разных классах смежности. Предположим, что существует такой класс смежности, который не будет описан с помощью $M_k$, тогда существует вектор, лежащий в этом классе, сумма которого некоторое целое число -- p.
Значит, этот вектор лежит в $M_p$, но классы смежности не пересекаются, пришли к противоречию. Таким образом, мы доказали, что разбиение на классы смежности верное, а значит и описание факторгруппы верно.

\subsection{Задача 5}
В качестве искомого гомоморфизма подойдёт $\varphi : \langle \mathds{Q}, + \rangle \rightarrow \langle \mathds{Q}[0, 1], \circ \rangle $, где $Q[0, 1) = \{ q | q \in \mathds{Q}$  $0 \leq q \leq 1 \}$, а операция описывается следующим образом:
$\forall A, b, \in \mathds{Q}[0, 1) \rightarrow a \circ b = \{ a + b\}$, т.е. это дробная часть от суммы двух чисел. Довольно очевидно, что $\langle \mathds{Q}[0, 1), \circ \rangle$ -- группа, ассоциативность следует из описания операции, нейтральный элемент это нуль, а 
для любого элемента $a \in \mathds{Q}[0, 1)$ $\exists \{ 1 - a\}$ (дробная часть взята, чтобы учесть случай $a = 0$). 
Осталось лишь описать само отображение:
\newline
 $\varphi : \forall x \in \mathds{Q} \rightarrow \varphi(x) = \{ x \}, obviously \{ x \} \in \mathds{Q}[0, 1)$
 
 Будем доказывать, что это отображение суть гомоморфизм: $\varphi (a + b) = \{ a + b \} = \{ [a] + \{ a \} + [b] + \{b \} \} = \{ [a] +[b] + [\{a \} + \{b \}] + \{ \{ a\} \{b\} \} \} = \{\{a\} + \{ b \} \} = \varphi(a) \circ \varphi(b)$, 
 следовательно, по определению это гомоморфизм, причем в силу такого отображения, очевидно, что образ элемента есть нуль тогда и только тогда, когда элемент принадлжеит множеству целых чисел, т.е. $\ker \varphi = \mathds{Z}$. Показали, что приведенный пример корректен.


 Докажем, что $\langle \mathds{Q}, + \rangle / \ker \varphi$ бексконечна, но каждый её элемент будет иметь конечный порядок. 
 Используем теорему о гомморфизме, т.к. $\varphi$ гомоморфизм, сл-но, $\langle \mathds{Q}, + \rangle / \ker \varphi \cong \varphi(\langle \mathds{Q}, + \rangle) = \langle \mathds{Q}[0, 1), \circ \rangle$, отсюда следует, что $\langle \mathds{Q}, + \rangle / \ker \varphi$ бексконечна, т.к. бесконечна группа $\langle \mathds{Q}[0, 1), \circ \rangle$, т.к. иначе бы не было биекции, т.е. изоморфизма.
 Аналогичными рассуждениями можно провести по поводя порядка элементов, т.е. пойдем от противного. Предположим, что порядок какого-то элемента из $a \in \langle \mathds{Q}, + \rangle / \ker \varphi$ бесконечен, тогда в силу свойства изоморфизма биективности, а также, что $\varphi (a^k) = (\varphi(a))^k$, где $k$ произвольное натуральное число, получим, что существует элемент в $\langle \mathds{Q}[0, 1), \circ \rangle$, который имеет бесконечный порядок, но любой элемент оттуда имеет конечный порядок,
 т.к. любая рациональная дробь $\frac{m}{n}$ на $[0, 1)$ имеет конечный знаменатель, тогдас сложив с помощью операции $\circ$ рассматриваемую дроб n раз, мы получим, нуль, т.е. порядок любой дроби из данной группы не превосходит знаменатель. 
 Отсюда получим противоречие, т.к. все элементы имеют конечный порядок.

\end{document}