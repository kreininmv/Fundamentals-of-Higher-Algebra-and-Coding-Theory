\documentclass[a4paper,14pt]{article} % тип документа
%\documentclass[14pt]{extreport}
\usepackage{extsizes} % Возможность сделать 14-й шрифт


\usepackage{geometry} % Простой способ задавать поля
\geometry{top=25mm}
\geometry{bottom=35mm}
\geometry{left=20mm}
\geometry{right=20mm}

\setcounter{section}{0}

%%%Библиотеки
%\usepackage[warn]{mathtext}
%\usepackage[T2A]{fontenc} % кодировка
\usepackage[utf8]{inputenc} % кодировка исходного текста
\usepackage[english,russian]{babel} % локализация и переносы
\usepackage{caption}
\usepackage{listings}
\usepackage{amsmath,amsfonts,amssymb,amsthm,mathtools}
\usepackage{wasysym}
\usepackage{graphicx}%Вставка картинок правильная
\usepackage{float}%"Плавающие" картинки
\usepackage{wrapfig}%Обтекание фигур (таблиц, картинок и прочего)
\usepackage{fancyhdr} %загрузим пакет
\usepackage{lscape}
\usepackage{xcolor}
\usepackage{dsfont}
%\usepackage{indentfirst}
\usepackage[normalem]{ulem}
\usepackage{hyperref}




%%% DRAGON STUFF
\usepackage{scalerel}
\usepackage{mathtools}

\DeclareMathOperator*{\myint}{\ThisStyle{\rotatebox{25}{$\SavedStyle\!\int\!\!\!$}}}

\DeclareMathOperator*{\myoint}{\ThisStyle{\rotatebox{25}{$\SavedStyle\!\oint\!\!\!$}}}

\usepackage{scalerel}
\usepackage{graphicx}
%%% END 

%%%Конец библиотек

%%%Настройка ссылок
\hypersetup { colorlinks=true, linkcolor=blue, filecolor=magenta, urlcolor=blue }
%%%Конец настройки ссылок

%%%Настройка колонтитулы
	\pagestyle{fancy}
	\fancyhead{}
	\fancyhead[L]{Домашнее задание}
	\fancyhead[R]{Крейнин Матвей, группа Б05-005}
	\fancyfoot{}
    \fancyfoot[C]{\thepage}
    \fancyfoot[R]{ОВАиТК}
%%%конец настройки колонтитулы

\begin{document}
%%%%Начало документа%%%%

\section{Задание 5}
\subsection{Задача 1}
Рассмотрим действие группы G на множестве X -- расрасок вершин пятиугольника 3 цветами.
Таким образом, искомое количество различных раскрасок из 3 цветов есть количество классов эквивалентности по отношению принадлежности элемента к орбите, так как два пятиугольника
совпадают наложением, тогда и только тогда, когда существует такой поворот из группы G, что пятиугольник переходит в рассматриваемый. 
Остается воспользоваться леммой Бернсайда: посчитаем $|X_g|$ для каждого элемента $g \in G$. Если рассматривать тождественное преобразование из G, то очевидно, что любая раскраска
 пятиугольника будет переходить в себя же, т.е. $|X|_e = 3^5$. Теперь рассмотрим повороты относительно центра, совпадение будет возможно лишь в случае, когда все вершины покрашены в один цвет, 
т.е. всего 3 пятиугольника. Последний случай с поворотами на $\pi$, чтобы поворот был осуществлен в этот же пятиугольник, необходима симметричность цветов относительно рассматриваемой оси, отсюда
конкретной оси будет $3^3$ совпадающих после поворота пятиугольников. Отсюда, применяя, лемму: $col = \frac{3^5 + 4 \cdot 3 + 5 \cdot 3^3}{10} = 39$

\underline{\textbf{Ответ: }} 39 раскрасок.

\end{document}